

\documentclass[twocolumn]{article}


%%%%%%%%%%%%%%%%%%%%%%%%%%%%%%%%%%%%%%%%%%%%%%%%%%%%%%%%%%%%%%%%%%%%%%%%%%%%%%%%%%%%%
% 								PACKAGE CONFIGURATION								%
%%%%%%%%%%%%%%%%%%%%%%%%%%%%%%%%%%%%%%%%%%%%%%%%%%%%%%%%%%%%%%%%%%%%%%%%%%%%%%%%%%%%%

%	For supporting accents (yep, my full name has two of them)
	\usepackage[utf8]{inputenc}
	\usepackage[T1]{fontenc}
	
%	Images
	\usepackage{graphicx}
	\fboxsep=0mm	%padding thickness
	\fboxrule=4pt	%border thickness
	\usepackage{xcolor}
	\usepackage{blindtext}

%	URLs
	\usepackage[hyphens]{url}
	\usepackage[hidelinks]{hyperref}
	\usepackage{hypcap}
	\hypersetup{breaklinks=true}
	\urlstyle{same}

%	Creating default inline TODOs
	\usepackage{blindtext}
	\usepackage[colorinlistoftodos,prependcaption]{todonotes}
	\usepackage{regexpatch}
	%\tracingxpatches%for debugging
	\makeatletter
	\xpatchcmd{\@todo}{\setkeys{todonotes}{#1}}{\setkeys{todonotes}{inline,#1}}{}{}
	\makeatother
	
%	Bibliography
	\usepackage[style=authoryear, backend=bibtex]{biblatex}
	\newcommand{\citep}[1]{\parencite{#1}}
	\addbibresource{bibliography.bib}
	\usepackage[nottoc]{tocbibind}
	\setlength\bibnamesep{2\itemsep}
	\setlength\bibitemsep{0.1\itemsep}
	\renewcommand*{\nameyeardelim}{\addcomma\space}
	\renewcommand*{\nonameyeardelim}{\addcomma\space}
	\DeclareLabeldate{\field{date}\field{eventdate} \field{origdate}\literal{nodate}}
	

	
%	Colors
	\usepackage{color}
	
%	Plotting
	\usepackage{tikz}
	\usepackage{pgfplots}
	\pgfplotsset{every axis/.append style={thick}}
	
%	Counting Text with TeXCount
	%	Needs to have perl installed and in the PATH: https://www.perl.org/get.html
	%	Needs the TeXCount perl script in the PATH: http://app.uio.no/ifi/texcount/intro.html


%%%%%%%%%%%%%%%%%%%%%%%%%%%%%%%%%%%%%%%%%%%%%%%%%%%%%%%%%%%%%%%%%%%%%%%%%%%%%%%%%%%%%
% 								COMMAND CONFIGURATION								%
%%%%%%%%%%%%%%%%%%%%%%%%%%%%%%%%%%%%%%%%%%%%%%%%%%%%%%%%%%%%%%%%%%%%%%%%%%%%%%%%%%%%%

%	Setting up my new commands
\newcommand*{\ContentPath}{./content}

\newcommand{\FutureContentNotes}[1]{{\color{blue}\textbf{\textit{Notes about future content}}.- {#1}}}

\newcommand{\FutureContentNotesHidden}[1]{}






%%%%%%%%%%%%%%%%%%%%%%%%%%%%%%%%%%%%%%%%%%%%%%%%%%%%%%%%%%%%%%%%%%%%%%%%%%%%%%%%%%%%%
% 										CONTENT										%
%%%%%%%%%%%%%%%%%%%%%%%%%%%%%%%%%%%%%%%%%%%%%%%%%%%%%%%%%%%%%%%%%%%%%%%%%%%%%%%%%%%%%


\title{Quake: The Rise and Decay of the Grandfather of eSports}
\author{Rubén Osorio López}

\begin{document}

\maketitle

\begin{abstract}

This paper takes a look at why and how Quake laid the first stone in the creation of the current version of eSports to then degrade in popularity to the point that it is currently. It will examine the different reasons that caused it to be the number one competitive eSport as well as analyse how those ceased to be an advantage over time. The study will relate to events that happened in the late 90s early 2000s (1997-2001), the period that encompasses the birth of Quake as an eSport, its rise to the top and the final decline at the hands of other competitive games like \textit{Counter-Strike}.

\end{abstract}

%	Here we include all the files that contain the different sections of the document
\section{Introduction}

Test text :P
\section{The History}
\label{sec::history}

\FutureContentNotes{
This section will mainly contain the events that happened between the year 1997 and the beginning of the new millennium. How and why these events happened and started sets a good framework to then go into detail in section \ref{sec::reasoning}.\\

Besides commenting on the facts about the events themselves in chronological order some extra comments analysing interesting phenomena will be added. A good example would be how, especially in the first few years, Quake was not actually "picked" as the game to become the eSport but it was originated from Quake fans wanting to compete so they created the tournaments, selecting which game will be in the tournaments was not even considered, they were very passionate Quake fans (and usually only played that game), so the choice was obvious (there really was nothing to choose).\\

In that era there were not factors like what game is the most marketable, how big is the fan-base or the hype, is it relevant for sponsors, do crowds want to watch it, can it get good venues or is this good for the future of my event. Most of those would be against Quake being chosen as a big eSports game right now, but then they were not relevant.\\

When commenting on the final years of Quake as the big eSports game we will mention how the biggest percentage of the public left the game to focus on others like \textit{Counter-Strike} \citep{game:cs} much in the same what that tournament organizers did (remarkably the CPL switching to \textit{Counter-Strike} \citep{game:cs} as the main game in the year 2000).\\

A possible approach is divide it by games: \textit{Quake}, \textit{Quake II} and \textit{Quake III}.\\

Some events that deserve mention are:
\begin{itemize}
	\item Red Annihilation
	\item All the CPL events (1997-2000) \citep{web:cpl}
	\item All QuakeCon events (merged once with CPL in 1998).
\end{itemize}

A good approach could be to note what came new with each game and relate how the games improved in each iteration and how the community also grew too the size and recognition of all the events that deserve to and will be mentioned.\\


}

\todo{Complete the History section}
\section{The Reasoning}
\label{sec::reasoning}


\FutureContentNotes{
This section will be the bread and butter component of the paper. Once the right context has been set in sections \ref{sec::introduction} and \ref{sec::history} it is possible to dive deeper into the reasons that support the statement of "Quake being the Grandfather of eSports".\\

One of the subsections will focus on how the design of the game affected its situation, how those made the game very compelling at the beginning and also how those ceased to be relevant in the latter years of Quake's success. Some interesting things that will be mentioned are:

\begin{itemize}
	\item How some of the most compelling mechanics were unintended.
	\item The very steep learning curve and high skill ceiling, considered elitist by many, and its implications (not fun for casual players, the charm of the game being inherently against casual, big eSports games and players).
	\item How the good quality multi-player broke through the market allowing people to play directly against other competitors.
	\item How community creativity was very encouraged and important (Custom servers and custom maps, a lot of modding freedom with the example of Goosman, that worked on Quake and the \textit{Counter-Strike}). Now developers controll the game much more as far as eSports and who makes what content.
	\item idSoftware as a company does not keep working on the same project, they like the make, release and move on from the game model.
\end{itemize}

Another very interesting and important section will be focused on the technology of the game. So much of its success depended on how breakthrough some of the features of the games were. Some of the things worth mention would be:

\begin{itemize}
	\item Real 3D engine: Moving and looking in all axis (adding up and down compared to Doom).
	\item Best Netcode to date (TCP/IP and servers).
	\item Realistic rendering techniques performing very well in real time for the era.
	\item Lack of other games with similar technology that would impress the potential public.
\end{itemize}


Some other situational advantages that it had at the beginning could also be mentioned. A good example would be that a huge amount of people claimed that they bough Quake only for the single-player since \textit{Doom} was already a huge success years before. Then when they had already played Quake's single-player mode and discovered the multi-player it opened this door of opportunity to keep enjoying the game in a whole different way. This is important since they did not market the game for the multi-player, and people did not feel like they were gambling and buying this "new thing", they were investing in a single-player game in their eyes from a company they already trusted.\\

}

With all the different events laid out a deeper analysis of why those happened is now feasible. The reasons had to be major to justify how started from nothing and went to become this new overwhelming force in the world of gaming.\\

But perhaps what could be more interesting is how the nature of those reasons that led Quake to its top position in the year 2000 quickly turned on it and ceased to be relevant given how the community had evolved and the new games that came to take Quake's spot at the zenith.\\

\subsection{Design: Skill-based Masterpiece}

A perfect example of what Caillois defined as a game of \textit{Agon} \citep{caillois1961man}, the game was quickly categorized as a skill-dependant game which required a ton of effort to perform. The next few paragraphs will focus on how it ended up being that way and, when applicable, why those design decision, mechanics or reasons are not compelling any more.

\subsubsection{Unintended but Important}

Kicking things off on a light note there are the unplanned mechanics and the features the game had which created completely unforeseen situations.\\

\textbf{Strafe-Jumping} was a movement technique in Quake that allowed the player to jump in a certain way that would increase his velocity past the theoretical limit in the game. If players chained together these jumps and started to bounce while moving incredibly fast around the map they would be using what was called \textbf{Bunny-Hopping}.\\

Both of those features were completely unintended and came from a bug in the movement related code yet they became so deeply important that the developers left said bug intact. The significance of the mechanic was very relevant to the deeply competitive community. It had a great impact on competitive gameplay while being very hard to learn and master. Rocket-Jumping\footnote{Rocket-Jumping is a technique in which the player shoots a rocket near its feet while jumping to boost their speed and height} is another example of this.\\

But mechanics were not the only unintended but important aspect of the game. The developers fantasized at one point about having big groups and clans of people competing in their game but they themselves did not expect that to materialize nor they intended to make it happen \citep{clanHistory}. As we now know, it did in fact happen and it was what kicked things off as far as the first small eSports tournaments.

\subsubsection{Elitism, prowess and expertise}

The main defining factor for Quake which still applies today. The game had and still has an incredibly \textbf{high skill ceiling} and \textbf{massively steep learning curve} caused by the very \textbf{hard to learn mechanics}, \textbf{aim and movement focused gameplay} and the use of mainly the \textbf{1v1 mode} in the competitive scene.\\

In the \textbf{late 90s} those factors were very appealing given the potential audience. In a time where competitive gaming was so small, the only \textbf{members of the community} were ones that now would be defined as "\textit{\textbf{hardcore gamers}}" or the "\textit{tryhards}". The core of the competitive gaming community that existed at the time found a game that focused on pure skill captivating.\\

But \textbf{the more a medium grows the more of its fan-base is composed by casual followers} instead of devoted aficionados. Such often called "\textit{casuals}" in a derogatory manner by the "\textit{hardcores}" are not as interested in getting into a game that will require months or even years of practice to be able to grasp its mechanics and be a contestant.\\

Currently, a \textbf{main focus of massive competitive games is to give that fun, fast and forgiving experience} to the casual players. That allows them to amass colossal fan-bases that feed the eSports machine.\\

Quake could not be more different. The aim, movement and strategy focused gameplay and the general difficulty create this situation in which even a game with two very evenly matched players often ends in utter and complete domination. A \textbf{hugely unbalanced final score} in this game does not mean that one player is significantly better, the \textbf{skill gap might actually be minimal}. Given this situation, imagine how \textbf{crushing} games can be to a new casual player who tries to compete with the old Quake veteran, the \textbf{experience would be demolishing}.\\

\FutureContentNotes{Insert here the graph with skill/win\%}



\subsection{Technology: Engineering Gem }

\subsection{Situational}


































\todo{Complete the Reasoning section}
\section{Conclusion}
\todo{Complete the Conclusion section}


%	And Here we have the bibliography citations
%	NOTE: Use \citep{id} to cite.
%\nocite{*} % Used to cite everything


\printbibliography[title={Bibliography},nottype=software] 
\printbibliography[title={Videogames},type=software]

%	Our In-Progress management things
%		\listoftodos
%		\todo{Remove the TODO list}

\end{document}







%
%	Useful build command to build and copy the pdf to another folder besides /src
%	It's probably for the best to restore the build commands to default before adding and using this one
%
%	txs:///pdflatex | xcopy .\*.pdf ..\pdf /sy
%