\section{The Reasoning}
\label{sec::reasoning}


\FutureContentNotes{
This section will be the bread and butter component of the paper. Once the right context has been set in sections \ref{sec::introduction} and \ref{sec::history} it is possible to dive deeper into the reasons that support the statement of "Quake being the Grandfather of eSports".\\

One of the subsections will focus on how the design of the game affected its situation, how those made the game very compelling at the beginning and also how those ceased to be relevant in the latter years of Quake's success. Some interesting things that will be mentioned are:

\begin{itemize}
	\item How some of the most compelling mechanics were unintended.
	\item The very steep learning curve and high skill ceiling, considered elitist by many, and its implications (not fun for casual players, the charm of the game being inherently against casual, big eSports games and players).
	\item How the good quality multi-player broke through the market allowing people to play directly against other competitors.
	\item How community creativity was very encouraged and important (Custom servers and custom maps, a lot of modding freedom with the example of Goosman, that worked on Quake and the \textit{Counter-Strike}). Now developers controll the game much more as far as eSports and who makes what content.
	\item idSoftware as a company does not keep working on the same project, they like the make, release and move on from the game model.
\end{itemize}

Another very interesting and important section will be focused on the technology of the game. So much of its success depended on how breakthrough some of the features of the games were. Some of the things worth mention would be:

\begin{itemize}
	\item Real 3D engine: Moving and looking in all axis (adding up and down compared to Doom).
	\item Best Netcode to date (TCP/IP and servers).
	\item Realistic rendering techniques performing very well in real time for the era.
	\item Lack of other games with similar technology that would impress the potential public.
\end{itemize}


Some other situational advantages that it had at the beginning could also be mentioned. A good example would be that a huge amount of people claimed that they bough Quake only for the single-player since \textit{Doom} was already a huge success years before. Then when they had already played Quake's single-player mode and discovered the multi-player it opened this door of opportunity to keep enjoying the game in a whole different way. This is important since they did not market the game for the multi-player, and people did not feel like they were gambling and buying this "new thing", they were investing in a single-player game in their eyes from a company they already trusted.\\

}

\todo{Complete the Reasoning section}