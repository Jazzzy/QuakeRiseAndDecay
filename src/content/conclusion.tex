\section{Conclusion}
\label{sec:conclusion}


\FutureContentNotesHidden{

To wrap up I would mention why after everything that has been explained I consider that Quake (or a Quake style game) can't be at the top of eSports again.\\

Besides that, it will contain some personal notes about the importance of this old and almost forgotten games. In some cases even unknown to new young players coming into eSports. After commenting on the monstrous growth that eSports had in the last 20 years going from nothing to the huge movement that it is, some credit should be given to games like Quake.\\

A curious like of thought could also be included about how when a medium gets big enough the things considered the best are not the most popular any more. There are a lot of interesting analogies, a good one would be with music. There was a time when bands could be at the same time considered the best and the most popular (Led Zeppelin?) but then Pop music was created. Pop (from popular) music is specifically designed to be, well, popular while not necessarily being the best. In the same way, there are games considered better than other games that are more popular in the eSports scene, and it is the case that the latter are better suited and designed to be popular, watchable, enjoyable for casual players, etc.\\

}

Much in the same way that the first computers, cars, televisions, movies or websites were pioneers and \textbf{disappeared from the public's mind}, Quake has suffered his own version of such fate. The very \textbf{first trailblazers} of a new medium or paradigm \textbf{are inherently modest} and concealed for the big masses. Nonetheless, that is \textbf{not a valid excuse to disregard their importance}. Sadly, there is a good chance that even new "\textit{hardcores}" coming into the word of eSports will not hear about Quake whatsoever.\\

Additionally, after considering everything contained in the previous sections, it is clear that \textbf{a pure Quake-Style game will not reign on top of the eSports behemoth again}. Such feat was only possible when eSports were more like a sprouting movement and not the gargantuan leviathan they are today.\\

As it happens with every growing type of media, \textbf{best and biggest tend to quickly diverge} as its popularity grows. Wolfgang Amadeus Mozart could be both the best and most popular musician in the XVIII century but if one tries to imply that the same could happen in the XXI century, Bruno Mars or Taylor Swift would beg to differ.\\

Striving to \textbf{make a popular eSports game} that is casual and sponsor friendly is now a science and \textbf{affects videogames' design}. The final version will most usually \textbf{drastically change} if the intention was to make the most \textbf{fair, challenging and skillful product}. In defiance of such trend we shall not be distressed. Quake, and games akin to the latter \textbf{will always have their place}. Fervent communities, although small, will continue to sustain them by reason of \textbf{the massive reward} of breaking their barrier to entry obtained in the form of even-handed entertainment, competitive enjoyment and vast sense of accomplishment.\\