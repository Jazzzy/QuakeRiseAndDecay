\section{The History}
\label{sec::history}


\FutureContentNotesHidden{
	This section will mainly contain the events that happened between the year 1997 and the beginning of the new millennium. How and why these events happened and started sets a good framework to then go into detail in section \ref{sec::reasoning}.\\
	
	Besides commenting on the facts about the events themselves in chronological order some extra comments analysing interesting phenomena will be added. A good example would be how, especially in the first few years, Quake was not actually "picked" as the game to become the eSport but it was originated from Quake fans wanting to compete so they created the tournaments, selecting which game will be in the tournaments was not even considered, they were very passionate Quake fans (and usually only played that game), so the choice was obvious (there really was nothing to choose).\\
	
	In that era there were not factors like what game is the most marketable, how big is the fan-base or the hype, is it relevant for sponsors, do crowds want to watch it, can it get good venues or is this good for the future of my event. Most of those would be against Quake being chosen as a big eSports game right now, but then they were not relevant.\\
	
	When commenting on the final years of Quake as the big eSports game we will mention how the biggest percentage of the public left the game to focus on others like \textit{Counter-Strike} \citep{game:cs} much in the same what that tournament organizers did (remarkably the CPL switching to \textit{Counter-Strike} \citep{game:cs} as the main game in the year 2000).\\
	
	A possible approach is divide it by games: \textit{Quake}, \textit{Quake II} and \textit{Quake III}.\\
	
	Some events that deserve mention are:
	\begin{itemize}
		\item Red Annihilation
		\item All the CPL events (1997-2000) \citep{web:cpl}
		\item All QuakeCon events (merged once with CPL in 1998).
	\end{itemize}
	
	A good approach could be to note what came new with each game and relate how the games improved in each iteration and how the community also grew too the size and recognition of all the events that deserve to and will be mentioned.\\
}

\subsection{1996-1997: The Early Days}

January 1996, two young players that identified themselves online as "\textit{Spleenripper}" and "\textit{Dr. Rigormortis}" were already building new systems and preparing a full LAN setup to get ready for the release of Quake \citep{clanHistory}. Said release was still half a year hence.\\

Quake was already a phenomenon within its own niche before it was even a tangible game you could play.  Although most of this knowledge comes from stories told by old players and currently unaccessible forums for the most part, there are still a few resources, mainly made by old Quake clans, that can give insight about how the situation was at the time \citep{clanHistory}.\\

The first relevant version of Quake that hit the public was \textit{QTest} \citep{qtest} on February 24, 1996. It became extremely popular amongst players that already knew about \textit{Doom} \cite{game:doom} and were eager to se the next big thing from \textit{idSoftware}. This version only could be played in multiplayer, which shows the enphasis that the devs put on that aspect of the game.\\

Soon after that it came the shareware edition of Quake. By this time the formation of clans such as \textit{The Amish}, \textit{Red Dragon} or \textit{Impulse 9} was already established in the community. The fact that these clans were astonishingly passionate about the game mixed with the gaming web boom at the time with some pages like \textit{Blue's News} hitting consistently 40.000 views a day. This created the perfect hotbed for the growth of such a new and fervent community.\\

Soon came the real deal, the commercial release of \textit{Quake} \citep{game:quake1} in May 1996 only heated the circumstances. Talks about creating tournaments were being held every day at the forums and some small ones started to happen, these were both small for current standards but big for the time. Also, the first QuakeCon \citep{quakecon} event was held in a hotel close to idSoftware's offices. It had 30 attendees in the first day and 100 by the end of the weekend once the news spread out.\\

At this time nothing could compare to what happened in May of 1997 when businesses like Intergraph, Microsoft and the developers from idSoftware got together to organize the biggest tournament to date, they called it \textbf{\textit{Red Annihilation}s}. Said event was hold during the now very famous \textbf{E3 expo} \cite{e3} in the famous \textbf{World Congress Center}.\\

This \textbf{one on one} tournament had more than 2000 participants qualifying online and the top 16 were flown to the live setting in the event to compete for \textbf{John D. Carmack's \footnote{John D. Carmack was the co-founder and lead developer in idSoftware during the era that concerns us.} Ferrari 328 GTS}. More and more breakthrough concepts kept being tied to this event. Not only the live audience was very significant but most of the spectators were able to watch the tournament via online in-game cameras professionally orchestrated. At the end of the tournament media like the NBC and The Wall Street Journal covered the event.\\

Right at that time the \textbf{CPL} \citep{web:cpl}, the pioneer in professional video-game tournament organizers, was created and a few months after, in October 1997 they organized their first event called \textbf{The FRAG} with a price pool consisting of 4000\$ in merchandise.\\

At the end of 1997, Quake was already becoming a big hit in the gaming community and it didn't show any signs of stopping.\\

\subsection{1998-Early 2000: Exponential growth}

\FutureContentNotes{This will contain the main QuakeCon and CPL events and their growth related to the growth of eSports and the Quake community as they were.}



\subsection{Late 2000-2001: Slayed by its own son}

\FutureContentNotes{This will contain the events happened near 2001 when the main and big tournament organizers left Quake to focus on new and more liked by the public games. Mainly Counter-Strike, a game that literally came from Quake.

We can talk about how Gooseman, who worked in Quake, made the first Counter-Strike, how CS used the Quake engine as a base and how it "stole" a lot of old Quake fans and communities that naturally switches games}
