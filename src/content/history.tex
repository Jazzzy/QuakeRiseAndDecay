\section{The History}
\label{sec::history}


\FutureContentNotesHidden{
	This section will mainly contain the events that happened between the year 1997 and the beginning of the new millennium. How and why these events happened and started sets a good framework to then go into detail in section \ref{sec::reasoning}.\\
	
	Besides commenting on the facts about the events themselves in chronological order some extra comments analysing interesting phenomena will be added. A good example would be how, especially in the first few years, Quake was not actually "picked" as the game to become the eSport but it was originated from Quake fans wanting to compete so they created the tournaments, selecting which game will be in the tournaments was not even considered, they were very passionate Quake fans (and usually only played that game), so the choice was obvious (there really was nothing to choose).\\
	
	In that era there were not factors like what game is the most marketable, how big is the fan-base or the hype, is it relevant for sponsors, do crowds want to watch it, can it get good venues or is this good for the future of my event. Most of those would be against Quake being chosen as a big eSports game right now, but then they were not relevant.\\
	
	When commenting on the final years of Quake as the big eSports game we will mention how the biggest percentage of the public left the game to focus on others like \textit{Counter-Strike} \citep{game:cs} much in the same what that tournament organizers did (remarkably the CPL switching to \textit{Counter-Strike} \citep{game:cs} as the main game in the year 2000).\\
	
	A possible approach is divide it by games: \textit{Quake}, \textit{Quake II} and \textit{Quake III}.\\
	
	Some events that deserve mention are:
	\begin{itemize}
		\item Red Annihilation
		\item All the CPL events (1997-2000) \citep{web:cpl}
		\item All QuakeCon events (merged once with CPL in 1998).
	\end{itemize}
	
	A good approach could be to note what came new with each game and relate how the games improved in each iteration and how the community also grew too the size and recognition of all the events that deserve to and will be mentioned.\\
}

\subsection{1996-1997: The Early Days}

January 1996, two young players that identified themselves on-line as "\textit{Spleenripper}" and "\textit{Dr. Rigormortis}" were already building new systems and preparing a full LAN set-up to get ready for the release of Quake \citep{clanHistory}. Said release was still half a year hence.\\

Quake was already a phenomenon within its own niche before it was even a tangible game you could play.  Although most of this knowledge comes from stories told by old players and currently unaccessible forums for the most part, there are still a few resources, mainly made by old Quake clans, that can give insight about how the situation was at the time \citep{clanHistory}.\\

The first relevant version of Quake that hit the public was \textit{QTest} \citep{qtest} on February 24, 1996. It became extremely popular amongst players that already knew about \textit{Doom} \citep{game:doom} and were eager to see the next big thing from \textit{id Software}. This version only could be played in multi-player, which shows the emphasis that the devs put on that aspect of the game.\\

Soon after that came the shareware edition of Quake. By this time the formation of clans such as \textit{The Amish}, \textit{Red Dragon} or \textit{Impulse 9} was already established in the community. The fact that these clans were astonishingly passionate about the game mixed with the gaming web boom at the time with some pages like \textit{Blue's News} hitting consistently 40.000 views a day. This created the perfect hotbed for the growth of such a new and fervent community.\\

Soon came the real deal, the commercial release of \textit{Quake} \citep{game:quake1} in May 1996 only heated the circumstances. Talks about creating tournaments were being held every day at the forums and some small ones started to happen, these were both small for current standards but big for the time. Also, the first QuakeCon \citep{quakecon} event was held in a hotel close to id Software's offices. It had 30 attendees in the first day and 100 by the end of the weekend once the news spread out.\\

At this time nothing could compare to what happened in May of 1997 when businesses like Intergraph, Microsoft and the developers from id Software got together to organize the biggest tournament to date, they called it \textbf{\textit{Red Annihilation}s}. Said event was held during the now very famous \textbf{E3 expo} \citep{e3} in the famous \textbf{World Congress Center}.\\

This \textbf{1v1} tournament had more than 2000 participants qualifying on-line and the top 16 were flown to the live setting in the event to compete for \textbf{John D. Carmack's \footnote{John D. Carmack was the co-founder and lead developer in id Software during the era that concerns us.} Ferrari 328 GTS}. More and more breakthrough concepts kept being tied to this event. Not only the live audience was very significant but most of the spectators were able to watch the tournament via online in-game cameras professionally orchestrated. At the end of the tournament media like the NBC and The Wall Street Journal covered the event.\\

Right at that time the \textbf{CPL} \citep{web:cpl}, the pioneer in professional video-game tournament organizers, was created and a few months after, in October 1997 they organized their first event called \textbf{The FRAG} with a prize pool consisting of \$4.000 in merchandise.\\

At the end of 1997, Quake was already becoming a big hit in the gaming community and it didn't show any signs of stopping.\\

\subsection{1998-Early 2000: Exponential growth}

\FutureContentNotesHidden{This will contain the main QuakeCon and CPL events and their growth related to the growth of eSports and the Quake community as they were.}

\textit{Quake II} \citep{game:quake2} was released at the very end of 1997 and quickly became the standard for tournament play. The short intervals and significant improvements between versions of Quake had the community permanently excited to learn and compete and this clearly showed during this era.\\

The \textbf{year 1998} fed on the previous success and saw a significant increase in both the size of big events and the amount of small ones. On July of this year the already mentioned CPL paired with some community members involved in previous tournaments organized the third QuakeCon event, which at the same time was the second FRAG event from CPL. At the time this presented a bad view to some members of the community which considered this as not the best event that could have been. Even in those circumstances it had an attendance of 800 people and 300 BYOC \footnote{\textbf{BYOC} stands for Bring Your Own Computer, the term is used for members of an event or a LAN party that carry and use their own machines.}. The prizes went from being merchandise to real money, giving \$1.000 for 5th place and scaling up to \$5.000 for first.\\

After contemplating the potential for big tournaments the \textbf{QuakeCon} organizers decided to dedicate more time to prepare the event without relying on the CPL. Going into the \textbf{year 1999} the event was much larger. The major involvement from id Software as a sponsor allowed to use a much bigger venue and have developers participate in the event. The attendance rose to 1100 people and 500 BYOC. Another important factor was the first ever tournament with \textit{Quake III} \citep{game:quake3}, which was still far from its release.\\ Later on in the year 2000, the event raised its numbers to 3000 attendees and 900 BYOC.\\

The \textbf{CPL} also kept blooming and establishing themselves as \textbf{the big fish in the growing pond} of eSports tournament organizers. They served as an example for many new game events and eSports leagues but none could compare to their success yet. After 1999 successful event they broke records when in the year 2000 they held a \textit{Quake III} tournament with a \textbf{prize pool of \$100.000}, 40.000 of which went to one of the rising stars of Quake, Johnathan ‘Fatal1ty’ Wendel. The existence of this characters only helped eSports and Quake to be more recognized and reach new potential fans.


\subsection{Late 2000-2001: Slayed by its own son}

\FutureContentNotesHidden{This will contain the events happened near 2001 when the main and big tournament organizers left Quake to focus on new and more liked by the public games. Mainly Counter-Strike, a game that literally came from Quake.

We can talk about how Gooseman, who worked in Quake, made the first Counter-Strike, how CS used the Quake engine as a base and how it "stole" a lot of old Quake fans and communities that naturally switches games}





\textbf{Minh "Gooseman" Le} \citep{gooseman} was a Vietnamese programmer deeply involved with the modding community of Quake. Him and another programmer in the same Quake modding team, \textbf{Jess Cliffe}, started to work on what would become the \textbf{heir of Quake} in the eSports scene, \textit{Counter-Strike} \citep{game:cs}.\\

\textit{\textbf{Counter-Strike}} came as a mod for \textit{Half-Life}, the incredibly successful First-Person Shooter game based on the \textit{Quake II} engine. Le was already used to work with said engine so modding \textit{Half-Life} felt familiar. This created a very interesting situation after the first version of the mod came in June 1999.\\

CS \footnote{CS is a common abbreviation for Counter-Strike.} kept gaining fans and getting bigger by the weeks. The story of Quake was being repeated but in a much shorter time frame. Old fans from the franchise were switching to CS, tournaments quickly put it in the same position as Quake in the year 2000 and the snowball just kept fattening and rolling down an increasingly steep hill.\\

Such was its success that \textbf{2001} saw the decline of Quake by the hands of a game that was a direct successor. Quake not only gave a large part of its technology and design to CS, but also a big part of its fan-base, Quake-based tournament and league organizers and, in general, a perfect platform for the next big eSports to grow.\\

A good example was what the CPL calls the beginning of their "\textbf{Golden Years}" \citep{web:cpl}, which at the time could be considered also the golden years of eSports given the position the CPL had representing them. In 2001, the event's main title was no other than CS replacing the long-standing Quake. This year saw a prize-pool of \$150.000 and was regarded like the biggest event to date.\\

\textbf{The leap was immense} and so the reasons to consider Quake as the main game for a tournament or league were becoming insignificant compared to the potential wins of \textbf{having CS instead}. Other new competitive games, sometimes from other growing genres, were quickly developing and feeding from the fan-bases of older games. A good example would be \textit{Starcraft} \citep{game:starcraft} representing RTSs \footnote{\textbf{RTS} is short for the Real-Time Strategy genre or games.}.\\

In general, given the deep relationship and obsession that the core of the Quake community had with the game, id Software's franchise fell to a stable and consistent position in the shadow of the biggest, more prominent games. Looking back during the beginning of the new millennium, it looked like the years of Quake ruling the early days of the eSports kingdom were long gone.













































